%# -*- coding: utf-8-unix -*-
\begin{overview}
\thispagestyle{empty}
%在2018年3月底,翻译\footnote{这个模板原本是用于一项书籍翻译计划的,关注我公众号的读者对此有所了解。然而由于版权原因,该译本无法公开分享。}进度已过大半,于是开始着手进行\LaTeX 排版。在此之前我对\LaTeX 的了解微乎其微,甚至第一次安装TexLive就出了问题,不得不重新安装。也是借着给这个译本排版的机会,才逐渐熟悉了这一软件的使用方法。
%
%如大家所见,模板的封面和扉页设计均高仿\footnote{李老师的书籍源码尚未公开,此为仿作。}自李文威老师《模形式初步》一书,并已得到李老师的使用许可;定理和定义环境则取材自网上流传的Elegantbook模版。我也从这一以模仿为主的学习过程中,对\LaTeX 有了更深入的了解。
%
%本模板命名为$\mathbb{ Q }$-book,谐音自cubic一词。由于是一个菜鸟的作品,自然还有许多瑕疵,对此模板的错误和不足之处还请各位多多包涵。


推荐参考文献-来自

电子教案地址:\href{https://github.com/zggl/AITeachingPlanDraft2020/}{人工智能基础——电子教案}
\end{overview}
